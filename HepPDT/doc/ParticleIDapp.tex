% this is latex
\documentclass[twoside,12pt]{article}
\usepackage{epsfig}
\setlength{\textheight}{230mm}
\setlength{\topmargin}{0mm}

%define page size: 
\setlength{\headheight}{0mm}
\setlength{\headsep}{0mm}
\setlength{\footskip}{5mm}
\setlength{\textwidth}{170mm}
\setlength{\oddsidemargin}{0mm}
\setlength{\evensidemargin}{0mm}

\raggedbottom

%line spacing
%%\renewcommand{\baselinestretch}{1.5}
\newcommand{\tstrut}{\protect\rule{0in}{0.2in}\hspace{1em}}
\def\etal{{\it et al.}}
\newcommand{\eg}{\textit{e.g.}}
\newcommand{\cpp}{C+\kern-0.35ex{}+}

%%%\input pslogos

\begin{document}

\section {ParticleID headers}
\label{PIDclass}

\subsection {ParticleID.hh}

\begin{tabbing}

{\bf namespace HepPDT} \\  \\

{\bf Free functions:} \\
\hspace{0.5in}  {\bf double spinitod( int js ); } \\
\hspace{0.5in}  {\bf int spindtoi( double spin ); } \\  \\

{\bf Public members:} \\
\hspace{0.5in}  {\bf enum location } 
       $\{$ nj=1, nq3, nq2, nq1, nl, nr, n, n8, n9, n10 $\}$; \\
\hspace{0.5in}  {\bf  struct Quarks } $\{$
    short nq1;
    short nq2;
    short nq3; $\}$; \\  \\

{\bf CLASS ParticleID } \\  \\

{\bf Public Methods:} \\
\hspace{0.5in}  {\bf ParticleID( int pid = 0 ); } \\
\hspace{0.5in}  The constructor.\\ \\
\hspace{0.5in}  {\bf ParticleID( const ParticleID \& orig ); } \\
\hspace{0.5in}  The copy constructor. \\ \\
\hspace{0.5in}  {\bf ParticleID \& operator=( const ParticleID \& ); } \\
\hspace{0.5in}  The assignment constructor. \\ \\
\hspace{0.5in}  {\bf void swap( ParticleID \& other ); } \\
\hspace{0.5in}  The swap constructor. \\ \\
\hspace{0.5in}  {\bf bool  operator $<$  ( ParticleID const \& other ) const;} \\
\hspace{0.5in}  Comparison operator. \\ \\
\hspace{0.5in}  {\bf bool  operator == ( ParticleID const \& other ) const;} \\
\hspace{0.5in}  Equality operator. \\ \\

\hspace{0.5in}  {\bf int    pid( )        const; } \\
\hspace{0.5in}  Returns the PID. \\ \\
\hspace{0.5in}  {\bf int abspid( )        const; } \\
\hspace{0.5in}  Returns the absolute value of the PID. \\ \\

\hspace{0.5in}  {\bf bool isValid( )   const; }\\
\hspace{0.5in}  Returns true if this integer obeys the numbering scheme rules. \\ \\
\hspace{0.5in}  {\bf bool isMeson( )   const; }\\
\hspace{0.5in}  Returns true if this integer obeys the meson  portion of the numbering scheme rules\\ \\
\hspace{0.5in}  {\bf bool isBaryon( )  const; }\\
\hspace{0.5in}  Returns true if this integer obeys the baryon portion of the numbering scheme rules.\\ \\
\hspace{0.5in}  {\bf bool isDiQuark( ) const; }\\
\hspace{0.5in}  Returns true if this integer obeys the diquark portion of the numbering scheme rules.\\ \\
\hspace{0.5in}  {\bf bool isHadron( )  const; }\\
\hspace{0.5in}  Returns true if either isBaryon or isMeson is true. \\ \\
\hspace{0.5in}  {\bf bool isLepton( )  const; }\\
\hspace{0.5in}  Returns true if the fundamentalID is 11-18. \\ \\
\hspace{0.5in}  {\bf bool isNucleus( )  const; }\\
\hspace{0.5in}  Returns true if this integer obeys the ion numbering scheme rules. \\ \\
\hspace{0.5in}  {\bf bool isPentaquark( )  const; }\\
\hspace{0.5in}  Returns true if this integer obeys the pentaquark numbering scheme rules. \\ \\

\hspace{0.5in}  {\bf bool hasUp( )      const; }\\
\hspace{0.5in}  Returns true if this is a valid PID and it has an up quark. \\ \\
\hspace{0.5in}  {\bf bool hasDown( )    const; }\\
\hspace{0.5in}  Returns true if this is a valid PID and it has a down quark.\\ \\
\hspace{0.5in}  {\bf bool hasStrange( ) const; }\\
\hspace{0.5in}  Returns true if this is a valid PID and it has a strange quark.\\ \\
\hspace{0.5in}  {\bf bool hasCharm( )   const; }\\
\hspace{0.5in}  Returns true if this is a valid PID and it has a charm quark.\\ \\
\hspace{0.5in}  {\bf bool hasBottom( )  const; }\\
\hspace{0.5in}  Returns true if this is a valid PID and it has a bottom quark.\\ \\
\hspace{0.5in}  {\bf bool hasTop( )     const; }\\
\hspace{0.5in}  Returns true if this is a valid PID and it has a top quark.\\ \\
\hspace{0.5in}  {\bf In practice, it is better to query the ParticleData class.} \\ \\

\hspace{0.5in}  {\bf int  jSpin( )        const; }\\
\hspace{0.5in}  jSpin returns 2J+1, where J is the total spin \\ \\
\hspace{0.5in}  {\bf int  sSpin( )        const; }\\
\hspace{0.5in}  sSpin returns 2S+1, where S is the spin \\ \\
\hspace{0.5in}  {\bf int  lSpin( )        const; }\\
\hspace{0.5in}  lSpin returns 2L+1, where L is the orbital angular momentum \\ \\
\hspace{0.5in}  {\bf int fundamentalID( ) const; }\\
\hspace{0.5in}  Returns the first 2 digits if this is a valid PID and it is neither
                neither a meson, a baryon, \\
\hspace{0.5in}	nor a diquark.  If this is a meson, baryon, or
		diquark, fundamentalID returns zero. \\ \\
\hspace{0.5in}  {\bf int extraBits( ) const; }\\
\hspace{0.5in}  Returns any digits beyond the 7th digit 
                (e.g. outside the numbering scheme). \\ \\
\hspace{0.5in}  {\bf Quarks quarks( ) const; }\\
\hspace{0.5in}  Returns a struct with the 3 quarks. \\ \\
\hspace{0.5in}  {\bf int threeCharge( ) const; }\\
\hspace{0.5in}  Returns 3 times the charge, as inferred from the quark content.\\
\hspace{0.5in}  If the fundamentalID is non-zero, then a lookup table is used. \\ \\
\hspace{0.5in}  {\bf int A( ) const; }\\
\hspace{0.5in}  If this is an ion, returns A.\\ \\
\hspace{0.5in}  {\bf int Z( ) const; }\\
\hspace{0.5in}  If this is an ion, returns Z.\\ \\
\hspace{0.5in}  {\bf unsigned short digit(location) const; }\\
\hspace{0.5in}  digit returns the base 10 digit at a named location in the PID \\ \\
\hspace{0.5in}  {\bf const std::string PDTname() const; }\\
\hspace{0.5in}  Returns the HepPDT standard name. \\ \\

{\bf Private Members:} \\
\hspace{0.5in}  {\bf int itsPID; } \\

\end{tabbing}

\vfill\eject

\subsection {ParticleIDTranslations.hh}

\begin{tabbing}

{\bf namespace HepPDT} \\  \\

{\bf Free functions:} \\
\hspace{0.5in} {\bf int translatePythiatoPDT( const int pythiaID );} \\
\hspace{0.5in} {\bf int translatePDTtoPythia( const int pid );} \\  \\

\hspace{0.5in} {\bf int translateIsajettoPDT( const int isajetID );} \\
\hspace{0.5in} {\bf int translatePDTtoIsajet( const int pid );} \\  \\

\hspace{0.5in} {\bf int translateHerwigtoPDT( const int herwigID);} \\
\hspace{0.5in} {\bf int translatePDTtoHerwig( const int pid );} \\  \\

\hspace{0.5in} {\bf int translateQQtoPDT( const int qqID);} \\
\hspace{0.5in} {\bf int translatePDTtoQQ( const int pid );} \\  \\

\hspace{0.5in} {\bf int translateGeanttoPDT( const int geantID);} \\
\hspace{0.5in} {\bf int translatePDTtoGeant( const int pid );} \\  \\

\hspace{0.5in} {\bf int translatePDGtabletoPDT( const int pdgID);} \\
\hspace{0.5in} {\bf int translatePDTtoPDGtable( const int pid );} \\  \\

\hspace{0.5in} {\bf int translateEvtGentoPDT( const int evtGenID );} \\
\hspace{0.5in} {\bf int translatePDTtoEvtGen( const int pid );} \\  \\

\end{tabbing}

\vfill\eject

\subsection {ParticleName.hh}

\begin{tabbing}

{\bf namespace HepPDT} \\  \\

{\bf Free functions:} \\

\hspace{0.5in} {\bf std::string  particleName( const int );} \\
\hspace{0.5in}  Returns the HepPDT standard name. \\ \\

\hspace{0.5in} {\bf void  listHepPDTParticleNames( std::ostream \& os );} \\
\hspace{0.5in} List all defined names.  \\ \\

\hspace{0.5in} {\bf bool validParticleName( const int );} \\
\hspace{0.5in} Verify that this particle ID has a valid name.  \\ \\

\hspace{0.5in} {\bf typedef  std::map$<$ int, std::string $>$  ParticleNameMap;} \\
\hspace{0.5in} {\bf ParticleNameMap const \&  getParticleNameMap();} \\
\hspace{0.5in} Access ParticleNameMap for other purposes.  \\
\end{tabbing}

Only getParticleNameMap is allowed to access ParticleNameMap.
ParticleNameMap is initalized by the first call to getParticleNameMap.
Because the map is static, this initialization only happens once.
We use a data table so that compile time is not impacted.

\vfill\eject

\end{document}
